\documentclass[12pt]{article}
\usepackage{natbib}
\usepackage{url}
\usepackage{html}
\usepackage{fullpage}
\newcommand{\hlink}{\htmladdnormallink}

\title{Zelig: Everyone's Statistical Software\thanks{The current
    version of this software is available at
    \texttt{http://gking.harvard.edu/zelig/}.}}
\author{Kosuke
  Imai\thanks{Department of Politics, Princeton
  University (Corwin Hall, Department of Politics, Princeton
  University, Princeton NJ 08544; \texttt{KImai@Princeton.Edu}).}
\and %
Gary King\thanks{Department of Government, Harvard
  University, (Center for Basic Research in the Social Sciences, 34
  Kirkland Street, Harvard University, Cambridge MA 02138; \texttt{King@Harvard.Edu}.}
\and %
Olivia Lau\thanks{Department of Government, Harvard University (Littauer Hall, North Yard, Cambridge MA 02138;
  \texttt{OLau@Fas.Harvard.Edu}).}}

\date{Version 1.0 \\\today}
   
\begin{document}
\maketitle

A growing proportion of statisticians and methodologists in political
science and other disciplines are converging on R, an extremely
powerful statistics package and programming language.  As an open
source project, R is freely accessible to anyone who wishes to use it
or contribute routines.  With thousands of contributors who have
created hundreds of packaged routines, R can deal with nearly any
statistical problem.  Although R's accessibility is one of its
strengths, it also reflects the enormous diversity in approaches to
statistic analysis across scholarly fields, and thus creates a virtual
babel of competing functions and inconsistent syntax.

To address these problems, we have created Zelig, a program that
simplifies R by creating a single, unified syntax for a large and
growing fraction of the models and methods that social scientists use.
Zelig literally \emph{is} ``everyone's statistical software'' because
Zelig's unified framework incorporates everyone else's (R) code.  We
hope it will \emph{become} ``everyone's statistical software'' for
applied data analysis.  Zelig makes R accessible for beginners, and
adds functionality to R which even advanced users may appreciate.
For many researchers, Zelig and R may well be the last statistical
tools they ever need to learn.

This short course introduces Zelig and R.  In addition to reviewing
command syntax and data manipulation procedures, the course
demonstrates practical applications in the context of applied
research in every subfield of political science.  These demonstrations
will include:
\begin{itemize}
\item Many standard statistical models, such as the bivariate logit,
  bivariate probit, exponential, gamma, logit, log-normal, least
  squares, multinomial logit, negative binomial, normal (Gaussian),
  ordinal logit, ordinal probit, Poisson, probit, rare events logit,
  and Weibull regression models;
\item Making statistical procedures easy to understand and interpret
  by calculating quantities of real interest, such as expected values,
  predicted values, first differences, risk ratios, and average
  treatment effects (following the logic of \hlink{Clarify}{http://gking.harvard.edu/stats.shtml\#clarify}; see
  \hlink{King, Tomz, and Wittenberg, 2000}{http://gking.harvard.edu/files/abs/making-abs.shtml}\nocite{KinTomWit00});
\item Bootstrapping to compute estimation uncertainty;
\item Presenting results using easy and flexible statistical graphics;
\item Running multiple analyses simultaneously by strata; 
\item Incorporating
  \hlink{MatchIt}{http://gking.harvard.edu/matchit}'s nonparametric
  matching methods as a preprocessing step for causal inference;
\item Performing conditional prediction and making in-sample causal inferences; 
\item Analyzing multiply imputed datasets created by
  \hlink{Amelia}{http://gking.harvard.edu/stats.shtml\#amelia} (see
  \hlink{King, Honaker, Joseph, and Scheve,
    2001}{http://gking.harvard.edu/files/abs/evil-abs.shtml})\nocite{KinHonJos01}
  and other programs; and
\item Outputting replication data sets and data structures so that you
  (and if you wish, anyone else) will be able to replicate the results
  of your analysis (see \hlink{King,
  1995}{http://gking.harvard.edu/files/abs/replication-abs.shtml}).\nocite{King95}
\end{itemize}

Zelig is a flexible program.  It can easily expand, and several
contributors are currently working to include their packages in Zelig.
As Zelig grows, you will have access to an increasing range of methods
and models.  We will conclude by explaining how you may add your
models and methods to the Zelig framework.

% \bibliographystyle{/home/king/c/gkbibtex/polisci}
% \bibliography{/home/king/c/gkbibtex/gk}
\bibliographystyle{polisci}
\bibliography{gk,gkpubs}

\end{document}

%%% Local Variables: 
%%% mode: latex
%%% TeX-master: t
%%% End: 

















